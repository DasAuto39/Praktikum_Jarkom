\section{Pendahuluan}
\subsection{Latar Belakang}
Dalam era digital saat ini, kebutuhan akan konektivitas jaringan semakin tinggi seiring dengan berkembangnya teknologi informasi. Penggunaan jaringan nirkabel (wireless) menjadi solusi yang semakin populer karena kemudahan instalasi, fleksibilitas, dan mobilitas yang ditawarkan. Dibandingkan jaringan kabel (wired), jaringan wireless memungkinkan perangkat untuk terhubung ke internet atau jaringan lokal tanpa harus dibatasi oleh jalur fisik. Hal ini sangat berguna dalam lingkungan perkantoran, kampus, rumah, hingga ruang publik.

Namun, di balik keunggulannya, jaringan wireless juga memiliki tantangan tersendiri seperti kecepatan yang bisa terpengaruh oleh jarak dan hambatan fisik, serta risiko keamanan yang lebih tinggi. Oleh karena itu, pemahaman yang baik tentang teknologi jaringan wireless, perangkat-perangkat yang digunakan, serta cara kerja dan keamanannya sangat penting, khususnya bagi mahasiswa atau praktisi di bidang teknologi informasi.

\subsection{Dasar Teori}
Jaringan wireless merupakan bentuk komunikasi data tanpa kabel yang menggunakan gelombang elektromagnetik, seperti gelombang radio atau inframerah, untuk mentransmisikan informasi antar perangkat. Salah satu teknologi wireless paling umum adalah Wi-Fi (Wireless Fidelity), yang memungkinkan perangkat seperti laptop dan smartphone terhubung ke jaringan tanpa menggunakan kabel fisik. Selain Wi-Fi, terdapat pula teknologi Bluetooth yang digunakan untuk komunikasi jarak dekat antar perangkat. Keunggulan utama jaringan wireless adalah mobilitas dan kemudahan instalasinya, meskipun memiliki keterbatasan seperti kecepatan yang bisa dipengaruhi jarak dan hambatan fisik, serta potensi risiko keamanan yang lebih tinggi dibandingkan jaringan kabel.

Dibandingkan dengan jaringan wired (berbasis kabel), jaringan wireless menawarkan fleksibilitas yang lebih besar. Jaringan kabel umumnya digunakan pada lingkungan yang membutuhkan koneksi stabil dan aman seperti laboratorium, data center, atau server, karena menawarkan kecepatan tinggi dan kestabilan transmisi data. Sebaliknya, jaringan wireless lebih sesuai digunakan pada perangkat mobile dan lingkungan yang dinamis seperti kampus, rumah, atau ruang publik.

Beberapa perangkat penting dalam jaringan wireless antara lain adalah modem, router, dan access point. Modem berfungsi menghubungkan jaringan lokal dengan penyedia layanan internet (ISP), router mengatur lalu lintas data dan dapat menyediakan koneksi Wi-Fi, sedangkan access point bertugas memancarkan sinyal nirkabel dan menghubungkan perangkat wireless ke jaringan kabel. Selain itu, perangkat seperti repeater dan wireless bridge juga berperan dalam memperluas jangkauan jaringan atau menghubungkan dua lokasi secara nirkabel.

Agar jaringan wireless tetap aman dan andal, digunakan berbagai protokol keamanan seperti WPA2 dan WPA3, serta teknik proteksi tambahan seperti firewall, VPN, dan autentikasi pengguna. Dengan memahami dasar-dasar teknologi wireless ini, pengguna dapat merancang dan mengelola jaringan nirkabel secara efektif sesuai kebutuhan.

%===========================================================%
\section{Tugas Pendahuluan}
\begin{enumerate}
 \item \textbf{Jelaskan apa yang lebih baik, jaringan wired atau jaringan wireless?}\\
    Keduanya memiliki kelebihan dan kekurangan masing-masing. Jaringan wired lebih baik dari segi kecepatan dan stabilitas, serta lebih aman karena akses fisiknya terbatas. Namun, jaringan wireless lebih unggul dalam hal mobilitas, kemudahan pemasangan, dan fleksibilitas. Pemilihan tergantung pada kebutuhan: jika dibutuhkan koneksi yang sangat stabil dan aman seperti untuk server, maka wired lebih cocok. Sedangkan untuk keperluan mobilitas dan kemudahan akses seperti di rumah, kampus, atau ruang publik, wireless lebih ideal.

    \item \textbf{Apa perbedaan antara router, access point, dan modem?}\\
    
    \begin{itemize}
    \item \textbf{Router} adalah perangkat yang mengatur lalu lintas data antar jaringan dan membagikan koneksi internet ke perangkat dalam jaringan lokal. Router juga dapat berfungsi sebagai pemancar sinyal Wi-Fi (jika jenisnya \textit{wireless router}).
    
    \item \textbf{Access Point (AP)} adalah perangkat yang menyediakan sinyal Wi-Fi dan menjadi jembatan antara jaringan kabel dengan perangkat \textit{wireless}. Biasanya digunakan untuk memperluas jangkauan jaringan nirkabel di suatu area.
    
    \item \textbf{Modem} adalah perangkat yang menghubungkan jaringan lokal dengan penyedia layanan internet (\textit{ISP}). Modem mengubah sinyal digital menjadi sinyal analog dan sebaliknya agar dapat dikirim melalui jalur telekomunikasi.
    \end{itemize}


    \item \textbf{Jika kamu diminta menghubungkan dua ruangan di gedung berbeda tanpa menggunakan kabel, perangkat apa yang kamu pilih? Jelaskan alasannya.}
    
    Saya akan memilih perangkat \textit{Point-to-Point (PtP) Wireless Bridge}, seperti \textit{AirGrid M5 HP}. Alasannya karena perangkat ini dirancang untuk menghubungkan dua lokasi secara \textit{wireless} dalam jarak jauh dengan kecepatan dan kestabilan tinggi. \textit{Wireless bridge} menggunakan antena arah yang fokus, sehingga cocok untuk menghubungkan dua bangunan tanpa harus menarik kabel jaringan secara fisik.
\end{enumerate}