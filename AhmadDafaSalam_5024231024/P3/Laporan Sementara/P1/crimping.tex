\section{Pendahuluan}
\subsection{Latar Belakang}
Praktikum ini bertujuan untuk  mempelajari konektivitas tanpa kabel dalam kehidupan sehari-hari, terutama di lingkungan yang menuntut fleksibilitas tinggi seperti kampus, kantor, dan tempat umum. Jaringan wireless memungkinkan perangkat saling terhubung tanpa kabel fisik, mempermudah mobilitas dan mempercepat akses informasi kapan saja dan di mana saja.

\subsection{Dasar Teori}
Jaringan wireless merupakan sistem komunikasi yang menggunakan gelombang elektromagnetik, seperti radio atau inframerah, untuk menghubungkan perangkat satu sama lain tanpa media kabel. Salah satu bentuk paling umum dari jaringan ini adalah Wi-Fi, yang bekerja berdasarkan standar IEEE 802.11 dan memanfaatkan sinyal radio untuk mentransmisikan data antara perangkat pengguna dengan access point. Selain Wi-Fi, terdapat pula Bluetooth yang digunakan untuk koneksi jarak dekat antar perangkat seperti speaker, headset, atau printer.

Dalam praktiknya, jaringan wireless memerlukan perangkat utama seperti access point, wireless router, dan wireless NIC (Network Interface Card). Access point berfungsi sebagai jembatan antara jaringan kabel dan nirkabel, sementara wireless router menggabungkan fungsi routing dan pemancar sinyal. Wireless NIC adalah komponen di perangkat seperti laptop dan smartphone yang memungkinkan mereka menangkap sinyal Wi-Fi. Penggunaan perangkat ini memungkinkan perangkat pengguna untuk bergabung dalam jaringan lokal tanpa koneksi fisik, meningkatkan kemudahan instalasi dan fleksibilitas.

Keamanan merupakan aspek penting dalam jaringan wireless karena data dikirim melalui udara yang rentan terhadap penyadapan. Untuk itu, protokol keamanan seperti WPA2 dan WPA3 digunakan untuk mengenkripsi data dan memastikan hanya perangkat yang sah dapat mengakses jaringan. Selain itu, identitas jaringan wireless dikenal dengan nama SSID (Service Set Identifier) yang wajib dikenali oleh perangkat sebelum dapat terhubung. Teknologi tambahan seperti repeater atau wireless bridge juga digunakan untuk memperluas jangkauan jaringan di area luas atau antar gedung tanpa kabel.

 \section{Tugas Pendahuluan}
\begin{enumerate}
    \item \textbf{Jelaskan apa yang lebih baik, jaringan wired atau jaringan wireless?}

    Keduanya memiliki keunggulan masing-masing tergantung kebutuhan. Jaringan wired (kabel) menawarkan koneksi yang stabil, cepat, dan aman, sehingga lebih cocok digunakan di lingkungan yang membutuhkan performa tinggi dan minim gangguan, seperti server, laboratorium, dan studio editing. Di sisi lain, jaringan wireless memberikan kemudahan dalam pemasangan serta fleksibilitas dalam mobilitas, sehingga lebih cocok untuk perangkat mobile seperti laptop dan smartphone. Jika dilihat dari segi fleksibilitas dan efisiensi penggunaan sehari-hari, jaringan wireless lebih unggul, tetapi untuk kestabilan dan kecepatan maksimum, jaringan kabel masih menjadi pilihan utama.

    \item \textbf{Apa perbedaan antara router, access point, dan modem?}

   {Router adalah perangkat yang mengatur lalu lintas data antar jaringan, misalnya antara jaringan lokal (LAN) dan internet. Mode} (modulator-demodulator) berfungsi mengubah sinyal digital menjadi sinyal analog (dan sebaliknya), dan biasanya digunakan untuk menghubungkan rumah atau kantor ke jaringan ISP. Sementara itu, Access Point (AP) adalah perangkat yang memperluas jaringan lokal dengan menyediakan koneksi nirkabel ke perangkat pengguna. Dalam praktiknya, modem menghubungkan ke internet, router mendistribusikan koneksi ke berbagai perangkat, dan access point memungkinkan perangkat wireless terhubung ke jaringan.

    \item \textbf{Jika kamu diminta menghubungkan dua ruangan di gedung berbeda tanpa menggunakan kabel, perangkat apa yang kamu pilih? Jelaskan alasannya.}

    Saya akan menggunakan perangkat point-to-point (PtP) wireless bridge, seperti Ubiquiti AirGrid atau NanoBeam. Perangkat ini memungkinkan dua titik saling terhubung secara nirkabel dalam jarak jauh dengan sinyal yang fokus dan stabil. Alasan pemilihan perangkat ini adalah karena PtP dirancang untuk menghubungkan dua lokasi yang berjauhan secara langsung tanpa kabel, dengan throughput yang cukup tinggi dan latensi rendah. Selain itu, instalasinya lebih murah dan cepat dibanding menarik kabel fiber antar gedung.
\end{enumerate}