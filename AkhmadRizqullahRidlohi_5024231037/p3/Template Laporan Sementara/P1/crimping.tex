\section{Pendahuluan}
\subsection{Latar Belakang}
Praktikum ini bertujuan untuk memberikan pemahaman tentang Wirelless LAN dan ubiquitous sehingga sebagai
mahasiswa dapat memahami dan mengimplementasikan teknologi ini dalam kehidupan sehari-hari. Wireless LAN adalah jaringan lokal yang menggunakan gelombang radio untuk mentransmisikan data,
sedangkan ubiquitous computing adalah konsep di mana teknologi komputer terintegrasi ke dalam lingkungan sehari-hari kita.

\subsection{Dasar Teori}
Jaringan wireless adalah jaringan yang menggunakan gelombang radio untuk mentransmisikan data tanpa menggunakan kabel.
Terdapat tipe jaringan wireless, yaitu bluetooth, dan Wi-Fi. Bluetooth adalah teknologi wireless yang digunakan untuk 
menghubungkan perangkat-perangkat dalam jarak dekat, sedangkan Wi-Fi adalah teknologi wireless yang digunakan untuk 
menghubungkan perangkat-perangkat dalam jarak yang lebih jauh. Bluetooth menggunakan frekuensi 2.4 GHz, sedangkan Wi-Fi
menggunakan gelombang radio untuk mengirim dan menerima data dengan kecepatan tinggi. Terdapat banyak standar Wi-Fi yang digunakan
contohnya IEEE 802.11a, IEEE 802.11b, IEEE 802.11g, dan IEEE 802.11n. Setiap standar memiliki kecepatan dan jangkauan yang berbeda-beda.
Teknologi wireless memiliki beberapa jenis arsitektur, yaitu Access Point (AP),Station(STA), SSID. Access Point adlaah 
arsitektur yang meneruskan data antara perangkat-perangkat yang terhubung ke jaringan wireless. 
Station adalah perangkat yang terhubung ke jaringan wireless seperti client dan wireless access point.
SSID adalah nama dari jaringan wireless yang digunakan untuk mengidentifikasi jaringan tersebut.
	
Tedapat beberapa beberapa perangkat yang digunakan dalam jaringan wireless, yaitu NIC, router, repeater, dan access point. 
Access Point adalah perangkat yang menghubungkan perangkat-perangkat dalam jaringan wireless dengan jaringan kabel.
Access Point dapat berfungsi sebagai bridge antara jaringan wireless dan jaringan kabel. Access point akan menciptakan
jaringan wireless lokal di area sekitarnya. Wireless router dapat berperan sebagai mengarahkan lalu lintas,megatur jalur 
pengiriman paket data, memancarkan sinyal wireless agar perangkat bisa terhubung tanpa kabel.
Wireless NIC adalah perangkat keras yang digunakan untuk menghubungkan perangkat ke jaringan wireless. Wireless NIC dapat berupa
kartu ekspansi yang dipasang di dalam perangkat atau perangkat eksternal yang terhubung ke perangkat melalui port USB.
Repeater adalah perangkat yang digunakan untuk memperluas jangkauan jaringan wireless dengan cara menerima sinyal dari access point.
 

%===========================================================%
\section{Tugas Pendahuluan}
\begin{enumerate}
	\item Tergantung kebetuhan jika ingin internet yang stabil dan cepat maka gunakan wired, tetapi jika ingin 
		menggunakan internet tanpa kabel maka gunakan wireless dan dapat digunakan dalam suatu radius.
	\item Moden adalah perangkat yang digunakan untuk menghubungkan jaringan lokal ke internet. Modem dapat mengubah sinyal digital
	menjadi sinyal analog sehingga dapat mengirimkan data melalui kabel telepon atau kabel coaxial. Modem juga dapat mengubah sinyal analog menjadi sinyal digital sehingga dapat diterima oleh perangkat-perangkat yang terhubung ke jaringan lokal.
	Router adalah perangkat yang digunakan untuk menghubungkan beberapa jaringan lokal menjadi satu jaringan yang lebih besar. Router dapat mengatur lalu lintas data antara jaringan-jaringan tersebut sehingga data dapat dikirimkan dengan efisien.
	Access point adalah perangkat yang memungkinkan perangkat terhubung ke jaringan kabel melalui jaringan nirkabel. Access point dapat memperluas jangkauan jaringan lokal sehingga perangkat-perangkat yang berada di luar jangkauan jaringan kabel dapat terhubung ke jaringan tersebut.
	\item Untuk menghubungkan dua ruangan di gedung yang berbeda dapat menggunakan wireless access point yang terhubung ke jaringan kabel. 
	Wireless access point dapat memperluas jangkauan jaringan lokal sehingga perangkat-perangkat yang berada di luar jangkauan jaringan kabel dapat terhubung ke jaringan tersebut. Wireless access point juga dapat digunakan untuk menghubungkan perangkat-perangkat yang berada di gedung yang berbeda dengan menggunakan koneksi internet.
\end{enumerate}