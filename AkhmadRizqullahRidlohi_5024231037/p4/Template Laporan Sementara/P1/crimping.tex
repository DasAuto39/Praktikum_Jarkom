\section{Pendahuluan}
\subsection{Latar Belakang}
Pada modul ini, praktikan akan memelajari bagaimana Firewall dan NAT, praktikan akan memahami apa itu 
Firewall dan NAT dan akan mengetahui bagaimana Firewall dan NAT bekerja pada aplikasinya.

\subsection{Dasar Teori}
Firewall adalah sistem keamanan jaringan yang berfungsi untuk memantau dan mengontrol lalu lintas 
jaringan masuk dan keluar berdasarkan aturan keamanan yang telah ditentukan. Tujuan utama firewall a
dalah untuk mencegah akses tidak sah ke atau dari jaringan pribadi. Terdapat beberapa jenis firewall 
yang digunakan sesuai kebutuhan dan skala infrastruktur jaringan. Packet filtering firewall adalah 
jenis paling dasar yang memeriksa paket data berdasarkan alamat IP sumber dan tujuan, port, dan 
protokol, namun tidak dapat melacak status koneksi. Stateful inspection firewall bekerja lebih 
canggih dengan memeriksa status dari setiap koneksi jaringan, sehingga mampu menentukan 
apakah paket merupakan bagian dari koneksi yang sah. Application layer firewall beroperasi pada 
lapisan aplikasi OSI dan mampu memfilter lalu lintas berdasarkan jenis aplikasi (seperti HTTP, FTP), 
memberikan tingkat keamanan yang lebih tinggi. Next-Generation Firewall (NGFW) menggabungkan fitur 
dari firewall tradisional dengan teknologi tambahan seperti inspeksi paket mendalam (DPI), pencegahan 
intrusi (IPS), dan kontrol aplikasi. Circuit level gateway memonitor sesi TCP dan UDP, serta 
memastikan bahwa sesi dimulai dengan benar sebelum memungkinkan data ditransfer tetapi circuit level 
gateway tidak bisa mengecek isi data yang dikirimkan.Software firewall yang berjalan di sistem operasi 
dan memberikan perlindungan untuk satu perangkat, serta hardware firewall yang berupa perangkat fisik 
khusus untuk mengamankan seluruh jaringan. Selain itu, terdapat cloud firewall, yaitu firewall 
berbasis cloud yang dikelola oleh penyedia layanan untuk melindungi infrastruktur cloud dan layanan 
berbasis internet. Cara kerja firewall secara umum melibatkan penyaringan lalu lintas berdasarkan 
aturan yang ditetapkan administrator jaringan; ketika lalu lintas jaringan diterima, firewall 
mengevaluasi paket terhadap aturan dan memutuskan apakah akan mengizinkan atau memblokir paket 
tersebut.

Network Address Translation (NAT) adalah teknik yang digunakan dalam jaringan komputer untuk mengubah 
alamat IP pada paket data saat mereka melewati router atau firewall. NAT memungkinkan beberapa 
perangkat dalam jaringan lokal menggunakan satu alamat IP publik yang sama untuk akses ke internet, sehingga 
menghemat penggunaan IP dan memberikan tingkat keamanan tambahan dengan menyembunyikan alamat aslinya. 
Cara kerja NAT adalah dengan mencatat setiap koneksi yang keluar dari jaringan lokal ke internet dan 
menggantikan alamat IP lokal dan/atau port sumber dengan alamat IP publik dan port tertentu, serta 
menyimpan informasi translasi tersebut di tabel NAT untuk memetakan respons yang kembali. Terdapat 
beberapa jenis NAT, yaitu: Static NAT yang menerjemahkan satu alamat IP lokal ke satu alamat IP 
publik secara permanen; Dynamic NAT yang memetakan alamat IP lokal ke alamat IP publik dari kumpulan 
alamat yang tersedia secara dinamis,dan Port Address Translation (PAT) yang 
memetakan banyak alamat IP lokal ke satu alamat IP publik dengan membedakan sesi berdasarkan nomor 
port.

Connection tracking adalah fitur pengamat lalu lintas jaringan, ia akan mencatat setiap koneksi yang terhubung
.Connection tracking melakukan manajemen trafik dengan cara menyimpan informasi penting dari koneksi yang 
terjadi yang dimana informasi tersebut akan digunakan untuk proses firewall filtering dan NAT.
%===========================================================%
\section{Tugas Pendahuluan}
Bagian ini berisi jawaban dari tugas pendahuluan yang telah anda kerjakan, beserta penjelasan dari jawaban tersebut
\begin{enumerate}
	\item Menggunakan Port Address Translation,dimana permintaan dari jaringan luar akan diproses dulu dalam 
	NAT, dan di NAT akan diolah sesuai port yang dituju dan nantinya akan diarahkan ke port tersebut. 
	\item Tergantung kebutuhan, jika ingin menggunakan internet yang aman maka prioritaskan firewall, dan jika
	ingin menggunakan IP publik yang sama dalam setiap jaringan lokal maka gunakanlah NAT (Modul Firewall dan NAT)
	\item Maka perangkat akan rentan terhadap serangan dari luar, karena tidak adanya pengecekan terlebih 
	dahulu sebelum pengiriman suatu data dan validasi koneksi {Modul Firewall dan NAT}. 
\end{enumerate}