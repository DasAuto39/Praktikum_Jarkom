\section{Pendahuluan}
\subsection{Latar Belakang}
Pada praktikum ini dilakukan percobaan VPN dan QOS, di mana VPN dan QOS adalah salah satu 
perihal yang umum pada konteks jaringan komputer, sehingga praktikum ini diperlukan agar praktikan 
mampu mengaplikasikan VPN dan QOS saat digunakan

\subsection{Dasar Teori}
Tunneling dalam jaringan komputer adalah teknik untuk mengirimkan data dari satu jaringan ke jaringan 
lain yang berbeda jenis melalui sebuah "terowongan digital". Konsep ini mirip seperti mengirim paket 
dari rumah A ke rumah B melewati berbagai jenis jalan—data dari satu perangkat dibungkus 
(encapsulated) menggunakan protokol tertentu agar bisa melewati jaringan perantara, dan kemudian 
dibuka kembali di tujuan. Pertama cara kerja tunneling, komputer pengirim membuat data yang ingin 
dikirimkan ke komputer penerima, lalu paket akan dimasukkan ke dalam bingkai ethernet dan dikirimkan 
ke router pengirim, lalu pada router pengirim, data dibungkus menggunakan format WAN, dan dikirim 
ke router penerima, lalu pada router penerima bungkus WAN dibuka dan paket dikirimkan ke komputer penerima.


Berbagai protokol tunneling digunakan tergantung pada kebutuhan. GRE (Generic Routing Encapsulation) 
digunakan untuk membungkus paket IP tanpa enkripsi. IPSec (Internet Protocol Security) menyediakan 
keamanan tinggi melalui enkripsi dan autentikasi, sering digunakan dalam koneksi VPN untuk melindungi 
data dari pihak ketiga. Protokol lain seperti PPTP, L2TP, dan SSTP juga digunakan untuk VPN, 
masing-masing dengan kelebihan dan keterbatasannya. SSH dapat digunakan untuk tunneling yang aman 
melalui koneksi remote, sementara VXLAN memungkinkan virtualisasi jaringan dalam skala besar seperti 
di data center atau lingkungan cloud.

Salah satu protokol yang paling penting dalam tunneling adalah IPSec. IPSec memberikan perlindungan 
data melalui fitur-fitur seperti autentikasi, enkripsi, integritas data, dan manajemen kunci. IPSec 
dapat bekerja dalam dua mode: Transport Mode dan Tunnel Mode, dengan perbedaan pada bagian paket IP 
yang dienkripsi. IPSec memanfaatkan protokol tambahan seperti ESP (Encapsulation Security Payload) 
dan AH (Authentication Header) untuk menjamin keamanan dan keaslian data.

MikroTik menyediakan dua metode utama yaitu Simple Queue dan Queue 
Tree. Simple Queue digunakan untuk pengaturan bandwidth per user atau per IP dengan cara yang mudah 
dan cepat, cocok untuk jaringan kecil. Sebaliknya, Queue Tree memberikan kontrol yang lebih kompleks 
dan fleksibel untuk jaringan besar dengan struktur bertingkat dan kemampuan pengelompokan trafik 
berdasarkan port, protokol, atau VLAN, namun memerlukan konfigurasi mangle terlebih dahulu. Kedua 
metode ini memungkinkan pengaturan prioritas trafik agar layanan penting seperti VPN dan video 
conference mendapat bandwidth lebih besar saat jaringan padat.

Pengaturan prioritas trafik sangat penting dalam menjaga kualitas layanan jaringan, terutama saat 
jaringan sedang padat. Dengan menggunakan fitur Quality of Service (QoS) atau sistem antrian seperti 
Queue Tree, administrator jaringan dapat memastikan bahwa trafik penting mendapat prioritas lebih 
tinggi dibanding aktivitas yang tidak mendesak seperti streaming atau download file besar. Dengan 
demikian, kinerja jaringan tetap optimal dan layanan penting tetap berjalan lancar.

%===========================================================%
\section{Tugas Pendahuluan}
Bagian ini berisi jawaban dari tugas pendahuluan yang telah anda kerjakan, beserta penjelasan dari jawaban tersebut
\begin{enumerate}
	\item 
	Untuk membangun koneksi yang aman antara kantor pusat dan kantor cabang sebuah perusahaan, VPN 
	(Virtual Private Network) jenis IPSec site-to-site menjadi solusi yang umum digunakan. Proses ini 
	dilakukan melalui dua fase utama, yaitu IKE (Internet Key Exchange) Phase 1 dan Phase 2. Pada 
	\textbf{IKE Phase 1}, kedua perangkat (router kantor pusat dan cabang) bertukar informasi guna 
	membentuk Secure Association (SA) awal menggunakan algoritma kriptografi yang disepakati. Fase 
	ini bertujuan untuk mengaeutentikasi identitas masing-masing pihak dan membuat tunnel aman untuk 
	pertukaran kunci. Parameter penting dalam fase ini mencakup algoritma enkripsi seperti AES-256, 
	metode autentikasi seperti pre-shared key (PSK) atau digital certificate, serta lifetime key, 
	misalnya 86400 detik (1 hari).

	Selanjutnya, \textbf{IKE Phase 2} digunakan untuk membuat SA kedua yang akan digunakan untuk 
	enkripsi dan dekripsi data aktual. Dalam fase ini, protokol ESP (Encapsulation Security Payload) 
	biasanya digunakan, dengan parameter keamanan seperti algoritma enkripsi (misalnya AES-128), 
	algoritma hashing (SHA-256), dan parameter lifetime yang lebih pendek, seperti 3600 detik (1 jam), 
	untuk meningkatkan keamanan. Kedua fase ini saling bergantung dan harus disesuaikan pada kedua 
	perangkat yang terlibat.

	Untuk konfigurasi sederhana pada router MikroTik untuk IPSec site-to-site, langkah-langkah 
	umum adalah sebagai berikut:

	\begin{itemize}
		\item Konfigurasi Phase 1 (Proposal, Policy, dan Peer): 
		\begin{verbatim}
		/ip ipsec proposal
		add name=ph1-proposal auth-algorithms=sha256 enc-algorithms=aes-256-cbc lifetime=1d

		/ip ipsec peer
		add address=REMOTE_IP exchange-mode=main secret=SharedKey name=branch-peer
		\end{verbatim}
		
		\item Konfigurasi Phase 2 (Policy):
		\begin{verbatim}
		/ip ipsec policy
		add src-address=192.168.1.0/24 dst-address=192.168.2.0/24 \
			sa-dst-address=REMOTE_IP sa-src-address=LOCAL_IP tunnel=yes \
			proposal=ph1-proposal
		\end{verbatim}
	\end{itemize}

	\textbf{Referensi}:
	\begin{itemize}
		\item MikroTik Documentation: \url{https://help.mikrotik.com/docs/display/ROS/IPsec}
\end{itemize}
	\item Untuk manajemen bandwidth di sebuah sekolah dengan total kapasitas 100 Mbps, digunakan 
	metode Queue Tree pada MikroTik RouterOS. Skema ini memungkinkan pembagian lalu lintas 
	berdasarkan jenis layanan, dengan pendekatan hierarchical queue. Pertama-tama, semua trafik 
	diberi parent queue dengan max-limit 100 Mbps. Kemudian dilakukan packet marking 
	pada masing-masing jenis trafik menggunakan fitur \texttt{/ip firewall mangle}. Trafik 
	ke e-learning dapat ditandai dengan \texttt{mark-packet=e-learning}, guru dan staf dengan 
	\texttt{mark-packet=guru-staf}, dan seterusnya. Berikut adalah Parent Queue dan Chold Queues:

\begin{itemize}
    \item Parent Queue:
    \begin{verbatim}
    /queue tree
    add name=Total parent=global max-limit=100M
    \end{verbatim}

    \item Child Queues:
    \begin{verbatim}
    add name=e-learning parent=Total packet-mark=e-learning max-limit=40M priority=1
    add name=guru-staf parent=Total packet-mark=guru-staf max-limit=30M priority=2
    add name=siswa parent=Total packet-mark=siswa max-limit=20M priority=3
    add name=cctv-sistem parent=Total packet-mark=cctv-sistem max-limit=10M priority=4
    \end{verbatim}
\end{itemize}

Konfigurasi mangle seperti berikut:
\begin{verbatim}
/ip firewall mangle
add chain=forward dst-address=192.168.10.0/24 action=mark-packet \
    new-packet-mark=e-learning passthrough=yes

add chain=forward dst-address=192.168.20.0/24 action=mark-packet \
    new-packet-mark=guru-staf passthrough=yes

add chain=forward dst-address=192.168.30.0/24 action=mark-packet \
    new-packet-mark=siswa passthrough=yes

add chain=forward dst-address=192.168.40.0/24 action=mark-packet \
    new-packet-mark=cctv-sistem passthrough=yes
\end{verbatim}


\textbf{Referensi}:
\begin{itemize}
    \item MikroTik Queue Tree Documentation: \url{https://help.mikrotik.com/docs/spaces/ROS/pages/328088/Queues}
\end{itemize}

\end{enumerate}

